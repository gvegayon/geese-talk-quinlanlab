% \PassOptionsToPackage{darkblue}{xcolor}
\documentclass[10pt,aspectratio=169]{beamer}
% \usecolortheme{dove}
\usetheme{Goettingen}
\usepackage[autocite=superscript]{biblatex}
\addbibresource{bibliography.bib}
% New notation ------------------------------------------------------------------
\newcommand{\expo}{E}
\newcommand{\Expo}{\mathbf{\expo{}}}
\newcommand{\expod}[1]{\expo_{#1}^D}
\newcommand{\expoc}[1]{\expo_{#1}^C}
\newcommand{\expoi}[1]{\expo_{#1}^I}

\newcommand{\covar}{\mathbf{x}}
\newcommand{\Covar}{\mathbf{X}}
\newcommand{\latent}{\mathbf{z}}
\newcommand{\Latent}{\mathbf{z}}

\newcommand{\Event}{\mbox{P}}
\newcommand{\pset}[1]{\mathcal{P}\left(#1\right)}
\newcommand{\size}[1]{\left|#1\right|}

\newcommand{\logit}[1]{\mbox{logit}^{-1}\left(#1\right)}
% -------------------------------------------------------------------------------

% -------------------------------------------------------------------------------

\usepackage{booktabs,tabularx}
\usepackage{fancybox}
\usepackage{ulem}
\input{notation-thesis.tex}

% Slide numbers
\newcounter{frame}[frame]
\setbeamertemplate{footline}[frame number] 

\graphicspath{{.}{fig/}}

\title[Predicting Gene Functions with Mech. ML]{Predicting of Gene Functions by Leveraging Biological Insights with Mechanistic Machine Learning\\\includegraphics[width=.25\linewidth]{DALL·E 2022-12-07 21.41.30}\vspace{-.5cm}}
\author[\hyperlink{https://ggv.cl}{https://ggv.cl}]{George G. Vega Yon, Ph.D.\\{\small \color{darkgray}george.vegayon@utah.edu}\vspace{-.5cm}}
\institute[UofUEpi]{Division of Epidemiology @ University of Utah}
\date{Dec 8th, 2022 @ Quinlan Lab\vspace{-.5cm}}

%%%%%%%%%%%%%%%%%%%%%%%%%%%%%%%%%%%%%%%%%%%%%%%%%%%%%%%%%%%%%%%%%%
%%%%%%%%%%%%%%%%%%%%%%%%%%%%%%%%%%%%%%%%%%%%%%%%%%%%%%%%%%%%%%%%%%
\newcommand{\toc}[0]{\begin{frame}{Table of Contents}
    \tableofcontents[current]
\end{frame}}

\newcommand{\tocsub}[0]{\begin{frame}{Table of Contents}
    \tableofcontents[currentsubsection]
\end{frame}}

\newcommand{\byside}[3]{\begin{minipage}[t]{#1\linewidth}%
		\bigskip
		\centering%
		\shadowbox{\Large #2}\hfill\\\bigskip%
		#3%
\end{minipage}}

%%%%%%%%%%%%%%%%%%%%%%%%%%%%%%%%%%%%%%%%%%%%%%%%%%%%%%%%%%%%%%%%%%
%%%%%%%%%%%%%%%%%%%%%%%%%%%%%%%%%%%%%%%%%%%%%%%%%%%%%%%%%%%%%%%%%%
\begin{document}

\begin{frame}
    \maketitle
    {\scriptsize Collaborators: Paul Thomas, Paul Marjoram, Huaiyu Mi, Christopher Williams (USC), Alun Thomas (UofU)}
\end{frame}

\begin{frame}{Table of Contents}
    \tableofcontents
    You can download the slides from \url{https://ggv.cl/quinlanlab2022}
    \uncover<2->{
    \vfill But first...}
\end{frame}

\begin{frame}{\texttt{whoami}}

\uncover<5->{\begin{center}
\begin{minipage}{.49\linewidth}\includegraphics[width=1\linewidth]{fig/hexlogos.png}\end{minipage}
\begin{minipage}{.49\linewidth}\includegraphics<2->[width=1\textwidth]{fig/github-stats-george.png}\end{minipage}
\end{center}}

\begin{itemize}[<+->]
    \item Research Assistant Professor in Epidemiology since Nov. 2021.
    \item A methodologist researching Statistical Computing applied to Complex Systems Modeling.
    \item Got a Ph.D. in Biostats from USC and an M.Sc. in Economics from Caltech.
    \item Working with networks + programming since $\sim$2010.
    \item Scientific software developer (R/C++).
\end{itemize}
    
\end{frame}

%%%%%%%%%%%%%%%%%%%%%%%%%%%%%%%%%%%%%%%%%%%%%%%%%%%%%%%%%%%%%%%%%%
%%%%%%%%%%%%%%%%%%%%%%%%%%%%%%%%%%%%%%%%%%%%%%%%%%%%%%%%%%%%%%%%%%
\section{Preliminaries}\toc{}

\begin{frame}<123123>
	\frametitle{Gene Function}
	
	Encode the synthesis of genetic products that ultimately are related to a
	particular aspect of life, for example
	
	\def\tmpwidth{.9\linewidth}
	
	\begin{table}
		\begin{tabular}{*{3}{m{.31\linewidth}<{\centering}}}
			\onslide<2->\bf Molecular function & %
			\onslide<3->\bf Cellular component & %
			\onslide<4->\bf Biological process \\
			\onslide<2->\href{http://amigo.geneontology.org/amigo/term/GO:0005215}{Active transport GO:0005215}& %
			\onslide<3->\href{http://amigo.geneontology.org/amigo/term/GO:0004016}{Mitochondria GO:0004016} & %
			\onslide<4->\href{http://amigo.geneontology.org/amigo/term/GO:0060047}{Heart contraction GO:0060047} \\
			\onslide<2->\includegraphics[width=\tmpwidth]{Sodium-potassium_pump_and_diffusion.png} & %
			\onslide<3->\includegraphics[width=\tmpwidth]{640px-Animal_Cell-svg.png} & % 
			\onslide<4->\includegraphics[width=\tmpwidth]{Systolevs_Diastole.png}
		\end{tabular}
	\end{table}
	
\end{frame}

\begin{frame}{Gene Function: the Gene Ontology Project}
    \begin{figure}
\includegraphics[width=.4\linewidth]{fig/go-logo.png}
\end{figure}

\begin{itemize}[<+->]
\item The GO project has $\sim$ 43,000 validated terms, $\sim$ 7.4M annotations on $\sim$ 5,200 species.
\item About $\sim$ 700,000 annotations are on human genes.
\item Only half of the human gene annotations are based on experimental evidence.
% \item Roughly half of human genes ($\sim$ 10,000 / 20,000) have some
% form of annotation.
% \item We know something of less than 10\% of known genes (near 1.7M).
\item About $\sim$ 173,000 publications have used the GO.% 40610
%\item An important effort of the GO has to do with phylogenetics...
\end{itemize}

\vfill
\hfill \small \textbf{source}: Statistics from \url{http://pantherdb.org/panther/summaryStats.jsp} and \url{http://geneontology.org/stats.html}\normalsize

\end{frame}

\begin{frame}{Predicting Gene Function: Current methods}
	
	Sequences, phylogenomics, and ML.
	
	\begin{itemize}[<+->]
		\item \textbf{BLAST}\autocite{altschulBasicLocalAlignment1990}: Prediction by sequence homology ($\sim$ 105,000 citations).
		\item \textbf{SIFTER}\autocite{Engelhardt2005, Engelhardt2011}: An evolutionary model of gene function/loss using phylogenetics.
		\item \textbf{aphylo}\autocite{VegaYon2021} (by yours truly): Another phylo-based method. Leverages negative annotations and pooled trees.
		\item \textbf{Phylo-PFP}\autocite{Jain2019}: A BLAST-based adding phylogenetic based distances.
		\item \textbf{DeepGOPlus}\autocite{kulmanovDeepGOPlusImprovedProtein2019}:One of the top-performing models in the literature, uses a 2D convolutional neural network on gene sequences.
		%		\begin{itemize}
			%			\item At most, three hidden layers (tried others)
			%			\item Typical activations relu and sigmoid
			%			\item Focus on feature engineering
			%			\item Beyond sequence similarity, there is \textbf{no biological theory supporting the model}.
			%		\end{itemize}
		\item \textbf{GOLabeler}\autocite{youGOLabelerImprovingSequencebased2018}: Top performing tool according to the CAFA challenge\autocite{Zhou2019cafa}, is an ensemble of various simple ML methods, including K-means and logistic regression.
            \item \textbf{DeepFRI}\autocite{gligorijevicStructurebasedProteinFunction2021}: Uses Graph Convolutional Neural Networks (GCNs) to predict function based on protein structure and genetic sequence.
	\end{itemize}
	\vfill \uncover<9->{None of the ML-based methods relies on biological theory (mechanistic models).}
\end{frame}

\section{Evolution of Gene Function}
\toc{}

\begin{frame}[c]
%	\frametitle{Uncovering the role of genes}
	
	\Large Is gene \textit{XYZ} involved in process \textit{ABC}?\normalsize\bigskip
	
	\begin{minipage}[t]{.33\linewidth}
		\centering
		\includegraphics[width=1\linewidth]{aphylo-data-0.png} \\
		Complex to directly assess
	\end{minipage}\hfill
	\uncover<2->{\begin{minipage}[t]{.33\linewidth}
		\centering
		\includegraphics[width=1\linewidth]{aphylo-data-1.png}\\
		But we may know from other species
	\end{minipage}}\hfill
	\uncover<3->{\begin{minipage}[t]{.33\linewidth}
		\centering
		\includegraphics[width=1\linewidth]{aphylo-data-2.png}\\
		And we further know how these are \textit{evolutionary} connected
	\end{minipage}}\hfill
	
% \bigskip\uncover<4>{\raggedleft\Large ... let's rephrase the question. \normalsize}

\end{frame}

\begin{frame}[c,label=aphylo-prob-diagram]
	\begin{center}
		\normalsize Is the human gene \textbf{XYZ} involved in process \textbf{ABC}, \uline{given what we know about that for other \textit{related} species}?
	\end{center}
	
	\begin{figure}
		\includegraphics[width=.9\linewidth]{aphylo-data-probability.pdf}
	\end{figure}
 % \pause
	% \Large \bigskip\hfill... Where is all this data?\normalsize


\end{frame}

\begin{frame}{Evolution of Gene function (of one function)}

Built a big model (lots of trees and annotations) called aphylo:

\begin{center}
\begin{minipage}{.44\linewidth}
    \begin{itemize}[<+->]
        \item Only two sources of data: Phylogenetic tree (\url{pantherdb.org}) and functional annotations (\url{geneontology.org}).
        \item Leverage negative annotation of GO terms (NOT).
        \item Use Felsenstein's tree pruning algorithm to compute tree likelihood.
        \item Fit pooled models featuring thousands of annotations in hundreds of trees (with split-second prediction capability).
    \end{itemize}
\end{minipage}
\begin{minipage}{.55\linewidth}
    \includegraphics[width=.8\linewidth, clip, trim={0 0 0 1.5cm}]{example-trees-good1-parts-1b.pdf}
\end{minipage}
\end{center}

\vfil\hfill\uncover<4->{... But what if we wanted to deal with multiple functions?}
  
\end{frame}

\begin{frame}{Evolution of Gene function (multiple functions)}
Tapping into Evol. Theory\\\bigskip

\begin{center}
\begin{minipage}{.44\linewidth}
\begin{itemize}[<+->]
    \item A fundamental part of Fun. Evol. is Duplication Events.
    \item Furthermore, knowing what happened to gene A (\textit{e.g.}, neofunctionalization) is highly informative to learn about the functional state of B.
    \item One way to model this is using a Markov Transition Model (as in SIFTER).
\end{itemize}
\end{minipage}
\begin{minipage}{.55\linewidth}
    \begin{figure}
    \centering
    \includegraphics[width=1\linewidth]{fig/Evolution_fate_duplicate_genes_-_vector.pdf}
    \caption{A key part of molecular innovation, gene duplication provides an opportunity for new functions to emerge (\href{https://en.wikipedia.org/wiki/File:Evolution_fate_duplicate_genes_-_vector.svg}{wikimedia})}
    \label{fig:duplication}
    \end{figure}
\end{minipage}
\end{center}

\end{frame}

% \begin{frame}[label=aphylo-current]
% 	\frametitle{Phylogenetics Modeling Strategies}
	
% 	\begin{minipage}[m]{.3\linewidth}
		
% 		\begin{figure}
% 			\includegraphics[width=.9\linewidth]{phylo-model-overview-legend.pdf}
% 		\end{figure}
		
% % 		\uncover<5->{\large SNA could help us with\\\alert{\textbf{Exponential\\Random~Graph Models}}}
% 	\end{minipage}\hfill
% 	\begin{minipage}[m]{.69\linewidth}
% 		\mode<beamer>{
% 			\begin{figure}
% 				\phantom{\includegraphics<1>[width=.9\linewidth]{phylo-model-overview-1.pdf}}%
% 				\includegraphics<2>[width=.9\linewidth]{phylo-model-overview-1.pdf}%
% 				\includegraphics<3>[width=.9\linewidth]{phylo-model-overview-2.pdf}%
% 				\includegraphics<4>[width=.9\linewidth]{phylo-model-overview.pdf}
% 			\end{figure}
% 		}
		
% 		\mode<handout>{
% 			\begin{figure}
% 				\includegraphics[width=.9\linewidth]{phylo-model-overview.pdf}
% 			\end{figure}
% 		}
% 	\end{minipage}
	
% \end{frame}

% ------------------------------------------------------------------------------
\begin{frame}[c]{Evolution of Gene function (multiple functions)}
	
	If we wanted to build a model with 3 functions, we would need to estimate...\\\bigskip
	\begin{minipage}[t]{.40\linewidth}
		\centering
		\shadowbox{Full Markov Transition Matrix}\\\bigskip
		\uncover<1->{\includegraphics[width=.8\linewidth]{aphylo-ergm-eq1.png} \\
			\vfill 
		}
	\end{minipage}\hfill
	\uncover<2->{\begin{minipage}[t]{.50\linewidth}
	    \vfill
	    \begin{itemize}
	        \item<2-> 512 parameters
	        \item<3-> Finding this many parameters is not easy.
	        \item<4-> Even if you can, interpretation is awkward.
	    \end{itemize}
	\end{minipage}}
	\vfill
\hfill\uncover<5->{Social Network Analysis may help us...}
\end{frame}



% -------------------------------------------------------------------------------
\begin{frame}[t]
	\frametitle{Exponential Random Graph Models (ERGMs)}
	\begin{center}
    \byside{.45}{Social Network}{%
	    \includegraphics[width=1\linewidth]{fig/adjmat-network.png}
	} \hfill %
	\begin{minipage}[t]{.5\linewidth}
	    \bigskip
	    \begin{itemize}
	    \item<2-> Not about individual ties.
	    \item<3-> Statistical inference on \textit{motifs} (triangles, dyads, homophily, etc.)
            \item<4-> Literature about ERGMs is vast, a.k.a. a low-hanging fruit.
	    \end{itemize}
	    \bigskip
	    \uncover<5->{Ultimately... \\
	    \Large
	    \textbf{ERGM} $\equiv$ \textbf{Modeling binary arrays}
	    \normalsize}
	    
	\end{minipage}
 \end{center}
	
\end{frame}

% -------------------------------------------------------------------------------
\begin{frame}[t]
	\frametitle{Exponential Random Graph Models (ERGMs)}
	\begin{center}
	\byside{.45}{Social Network}{%
	    \includegraphics[width=1\linewidth]{fig/adjmat-network.png}
	} \hfill %
	\byside{.45}{Evolutionary Event}{%
	    \includegraphics[width=1\linewidth]{fig/adjmat-aphylo.png}
	}
        \end{center}
	\vfill
	Social Networks are usually represented as \textbf{adjacency matrices}, and so can evolutionary events!
	
\end{frame}

% ------------------------------------------------------------------------------
\begin{frame}[c]{Evolution of Gene function (multiple functions) (cont.)}
	
	If we wanted to build a model with 3 functions, we would need to estimate...\\\bigskip
	
	\begin{minipage}[t]{.40\linewidth}
		\centering
		\shadowbox{Full Markov Transition Matrix}\\\bigskip
		\includegraphics[width=.8\linewidth]{aphylo-ergm-eq1.png} \\
			\vfill 512 parameters
	\end{minipage}\hfill
	\begin{minipage}[t]{.19\linewidth}
		\centering 
		\uncover<3->{
			\includegraphics[width=.8\linewidth]{aphylo-ergm-eq2.png}\\
			Easier to fit \\
			Easier to interpret}	
	\end{minipage}\hfill
	\uncover<2->{\begin{minipage}[t]{.40\linewidth}		
		\centering
		\shadowbox{Sufficient statistics}\\\bigskip
		\includegraphics[width=.8\linewidth]{aphylo-ergm-eq2.png} \\
			\vfill 11 parameters (for example)
	\end{minipage}}
\end{frame}



\begin{frame}
\begin{minipage}[t]{.95\linewidth}
\small
	\def\fwidth{.55\linewidth}
	\begin{table}
	\begin{tabular}{m{.14\linewidth}<\centering m{.25\linewidth}m{.46\linewidth}}
	\toprule
	Rep. & Description & Definition  \\ \midrule
	\includegraphics[width=\fwidth]{fig/term-gain.png} & %
		Gain of function & $(1 - x_p)\sum_{n:n\in Off}x_n$  \\
	\includegraphics[width=\fwidth]{fig/term-loss.png} & %
		Loss of function & $x_p\sum_{n:n\in Off}(1 - x_n)$  \\
	\includegraphics[width=\fwidth]{fig/term-subfun.png} & %
		Subfunctionalization & $x_p^kx_p^j\sum_{n\neq m}x_n^k(1-x_n^j)(1-x_m^k)x_m^j$  \\
	\includegraphics[width=\fwidth]{fig/term-neofun.png} & %
		Neofunctionalization & $x_p^k(1 - x_p^j)\sum_{n\neq m}x_n^k(1-x_n^j)(1-x_m^k)x_m^j$ \\
	\includegraphics[width=\fwidth]{fig/term-longest.png} & %
		Longest branch gains & $(1-x_p^k)\isone{x_m^k : m=\text{argmax}_n\text{blength}_n}$ \\
	\bottomrule
	\end{tabular}
	\caption{Example of sufficient statistics for evolutionary transitions.}
	\end{table}
 \normalsize
 \end{minipage}
 \begin{minipage}[t]{.04\linewidth}\hspace{-3cm}
    \includegraphics[width=6\linewidth]{fig/Evolution_fate_duplicate_genes_-_vector.pdf}\vspace{-3cm}
\end{minipage}
\end{frame}

\begin{frame}{GEESE: \textbf{GE}ne functional \textbf{E}volution using \textbf{S}ufici\textbf{E}ncy}

I implemented what I just described in a C++ library with a companion R package called geese. The question is: How much do we earn by using these motifs?\\\bigskip
\pause
\begin{itemize}[<+->]
    \item Using 37 phylogenetic trees featuring 401 go annotations.
    \item \textbf{aphylo}: Fitted a \textit{simple gain/loss} of function model.
    \item \textbf{GEESE}: Fitted an evolutionary model controlling for \textit{functional preservation} (i.e., like neofun or subfun.)
    \item Fitted both of them using MCMC.
    \item Used LOO cross-validation to compute aggregated AUCs and MAE.
\end{itemize}
\end{frame}

\begin{frame}{GEESE for predicting gene function (cont.)}

How much can we gain from a joint dist. model?
    
\begin{figure}
\centering
    \includegraphics[width = .7\linewidth]{fig/mcmc-analysis-unif-prior-curated-auc-and-mae.pdf}
    % \caption{Caption}
    \label{fig:auc-geese-vs-aphylo}
\end{figure}

Just controlling for preservation (having only one duplicate changing) significantly improves our predictions.
    
\end{frame}




%%%%%%%%%%%%%%%%%%%%%%%%%%%%%%%%%%%%%%%%%%%%%%%%%%%%%%%%%%%%%%%%%%
%%%%%%%%%%%%%%%%%%%%%%%%%%%%%%%%%%%%%%%%%%%%%%%%%%%%%%%%%%%%%%%%%%
\section{Mechanistic Machine Learning}\toc{}

\begin{frame}{Mechanistic Machine Learning: State-of-the-art}

	\begin{itemize}
		\item After all the data pouring, attention to causal inference and mechanistic models is coming back\autocite{bakerMechanisticModelsMachine2018, pearlSevenToolsCausal2019}
		\item Applications in Physics, Chemistry, Biomedical Imaging, and Biology\autocite{willardIntegratingScientificKnowledge2022a, jornerMachineLearningMeets2021, gawIntegrationMachineLearning2019, altaweraqiImprovedPredictionGene2022} show the benefits of combining the two approaches.
	\end{itemize}

\begin{center}
	\byside{.4}{Mechanistic Models}{
		\begin{itemize}
			\item Inference driven (causality)
			\item Great for small datasets
			\item Not the most accurate
		\end{itemize}
	}
	\byside{.4}{Machine Learning Models}{
		\begin{itemize}
			\item Data-driven (prediction)
			\item Lots of points to ``learn''
			\item Great for big data
		\end{itemize}
	}
\end{center}

\vfill

\alert{Important}: Mechanistic Machine Learning \textbf{is not} domain-knowledge aided feature engineering. You need a whole other model to complement the ML algorithm.

\end{frame}

\begin{frame}{Potential Strategies}
	
	\begin{itemize}
	\item Use machine learning to learn the errors of a mechanistic model.
	\item Add constraints to the ML algorithm based on a mechanistic model.
	\item Add mechanistic predictions as a feature of a machine learning model.
	\end{itemize}

	\begin{figure}
	\centering
	\includegraphics[width=.3\linewidth]{fig/DALL·E 2022-12-07 21.41.30.png}
	\caption{``A van Gogh-style painting of an android holding a large biology book in one hand and a computer in another, examining an evolutionary tree that, instead of leaves, have genes.''--DALL-E's interpretation of my description \hyperlink{https://labs.openai.com/s/s0GoDQ64OMRfMr1y6uRXtmo9}{(link)}}
\end{figure}
\end{frame}

\begin{frame}
\begin{figure}
    \centering
    \includegraphics[width=.7\linewidth]{fig/Baker et al (Biol. Lett., 2018)-fig1.pdf}
    \caption{``\textit{The inputs and outputs from machine learning and mechanistic modelling approaches, and the potential for synergy between the two.}'' (Figure 1 reproduced directly from \textcite{bakerMechanisticModelsMachine2018}.}
    \label{fig:baker}
\end{figure}
\end{frame}

\begin{frame}{}
    \begin{figure}
        \centering
        \includegraphics[width=.8\linewidth]{fig/Rana et al (Current Opinion in Biotechnology, 2020)-fig1.pdf}
        \caption{\footnotesize ``\textit{\textbf{Top}: Machine learning is applied to the input data to identify the important features for constructing reduced order constraint-based models; the CBM simulations can be iteratively matched with input data for convergence until the proper set of features are identified. \textbf{Bottom}: Machine learning is iteratively applied to CBM simulations to reconcile with experimental data. Interplay between the Top and Bottom parts can iteratively lead to convergence between CBM simulations, experimental data and machine learning based predictions.}'' (Figure 2 reproduced directly from \textcite{ranaRecentAdvancesConstraintbased2020})}
        \label{fig:rana}
    \end{figure}
\end{frame}

\begin{frame}
    \begin{figure}
        \centering
        \includegraphics[width=1\linewidth]{Jorner et al (Chemical Science, 2021)-fig2.pdf}
        \caption{``\textit{Different types of quantitative reaction prediction approaches. Mechanistic DFT (a) and QSRR (b) are the current gold standard methods. Deep learning models (c) are emerging as an alternative. Hybrid models (d) combine mechanistic DFT modelling with traditional QSRR}'' (Figure 2 reproduced directly from \textcite{jornerMachineLearningMeets2021}}
        \label{fig:jorner}
    \end{figure}
\end{frame}

%%%%%%%%%%%%%%%%%%%%%%%%%%%%%%%%%%%%%%%%%%%%%%%%%%%%%%%%%%%%%%%%%%
%%%%%%%%%%%%%%%%%%%%%%%%%%%%%%%%%%%%%%%%%%%%%%%%%%%%%%%%%%%%%%%%%%
\section{Proof of Concept}\toc{}
%%%%%%%%%%%%%%%%%%%%%%%%%%%%%%%%%%%%%%%%%%%%%%%%%%%%%%%%%%%%%%%%%%
%%%%%%%%%%%%%%%%%%%%%%%%%%%%%%%%%%%%%%%%%%%%%%%%%%%%%%%%%%%%%%%%%%

\begin{frame}{Beyond GO and Trees... Bgee}


The \includegraphics[width=.1\linewidth]{fig/bgee_logo.png} project ``is a \textbf{database} for retrieval and \textbf{comparison of gene expression} patterns \textbf{across multiple animal species}. It provides an intuitive answer to the question `where is a gene expressed?' and supports research in cancer and agriculture as well as evolutionary biology.'' -- \textcite{bastianBgeeSuiteIntegrated2021}

\begin{itemize}
    \item Raw expression annotations.
    \item Standardized expression scores (so can compare across species/tissues).
    \item And also yes/no expression annotations based on the standardized scores.
\end{itemize}

\end{frame}

\begin{frame}{Beyond GO and Trees... Bgee (cont.)}

Divergence across species in gene expression levels has been linked to evolutionary events\autocite{nabholzHighLevelsGene2013, hodgins-davisEvolvingGeneExpression2009}, \textit{i.e.}, expression levels clustered phylogenies.\\\bigskip

Thinking of different ways to use it as:\\\bigskip

\begin{itemize}
    \item As an additional feature for our model: ``Given the phylo, observed annotations, \textbf{and expression levels in $n$ tissues}, \dots''
    \item As 0/1 variable (expression is present/absent) to predict in our model: ``Model the evolution of gene function \textbf{and expression}.''
    \item As part of a prediction model in, say, a Machine Learning Model.
\end{itemize}

    
\end{frame}


%%%%%%%%%%%%%%%%%%%%%%%%%%%%%%%%%%%%%%%%%%%%%%%%%%%%%%%%%%%%%%%%%%
%%%%%%%%%%%%%%%%%%%%%%%%%%%%%%%%%%%%%%%%%%%%%%%%%%%%%%%%%%%%%%%%%%

\begin{frame}[t]{What went into the blender}
\small
\begin{center}
    \uncover<2->{\byside{.45}{Data Feats}{
        \begin{itemize}
            \item Bgee 15 dataset: approx 7 billion annotations for 1.5 million genes.
            \item Our dataset: 1,484 predictions for 1,318 genes.
            \item Search by Gene name: 9,923,427 Bgee annotations.
        \end{itemize}
    }}
    \uncover<3->{\byside{.45}{Final model}{
    \begin{itemize}
        \item 10 GO terms (in a full-Markov model, this is 1 MM params).
        \item 278 annotations for 256 genes.
        \item 10 GEESE predictions for each gene.
        \item 46 Bgee score for gene expression computed as \textbf{\textit{mean expression score by gene-genus}}
    \end{itemize}
    }}
\end{center}

\vfill
\raggedright
\scriptsize \uncover<4->{\textbf{GO terms}: GO:0004672, GO:0004713, GO:0004867, GO:0005730, GO:0005829, GO:0005886, GO:0006468, GO:0009408, GO:0015020, GO:0060070

\textbf{Genus}: Anguilla, Anolis, Astatotilapia, Astyanax, Bos, Branchiostoma, Caenorhabditis, Callithrix, Canis, Capra, Cavia, Cercocebus, Chlorocebus, Danio, Drosophila, Equus, Esox, Felis, Gadus, Gallus, Gasterosteus, Gorilla, Heterocephalus, Homo, Latimeria, Lepisosteus, Macaca, Manis, Meleagris, Microcebus, Monodelphis, Mus, Neolamprologus, Nothobranchius, Ornithorhynchus, Oryctolagus, Oryzias, Ovis, Pan, Papio, Poecilia, Rattus, Salmo, Scophthalmus, Sus, Xenopus}
\normalsize
\end{frame}

\begin{frame}{Mechanistic ML}

We are comparing three models:

\begin{center}
    \uncover<1->{\byside{.3}{GEESE}{Phylogenetic based predictions (evolution of gene function)}}
    \uncover<2->{\byside{.3}{Bgee}{Linear Prob. model using expression as predictors.}}
    \uncover<3->{\byside{.3}{GEESE + Bgee}{Linear Prob. model using expression as predictors \textbf{and} predictions made by GEESE.}}
\end{center}
    
\end{frame}

\begin{frame}{Mechanistic ML (prelim res.)}
    \begin{figure}
        \centering
        \includegraphics[width = .7\linewidth]{fig/logit-aucs-ols-geese.pdf}
        % \caption{Prelimi results 1}
        \label{fig:auc-geese-plus-bgee}
    \end{figure}
\vfill
\small Both AUC and MAE were computed only using predictions for which we knew the true value.\normalsize
\end{frame}


%%%%%%%%%%%%%%%%%%%%%%%%%%%%%%%%%%%%%%%%%%%%%%%%%%%%%%%%%%%%%%%%%%
%%%%%%%%%%%%%%%%%%%%%%%%%%%%%%%%%%%%%%%%%%%%%%%%%%%%%%%%%%%%%%%%%%
\begin{frame}{Discussion}
\uncover<2->{\byside{.32}{\normalsize Gene function}{\small\begin{itemize}
        \item We are racing to discover what genes do.
        \item Experimental assessment is expensive (money and time,) $\to$ automatic annotations.
        \item Many ways to do it (seq. homology, evolutionary theory, ML, etc.)
        \item The best methods use ML (pattern discovery)... but none (AFAIK) are based on bio. theory.
\end{itemize}}}
\uncover<3->{\byside{.32}{\normalsize Evol. Model}{\small\begin{itemize}
    \item We proposed an Evolutionary model of Gene Function.
    \item This new model, GEESE, uses sufficiency to reduce ``Markov complexity.''
    \item We showed it really helps.
\end{itemize}}}
\uncover<4->{\byside{.32}{\normalsize Mechanistic ML}{\small
\begin{itemize}
    \item Mechanistic Machine Learning (mixing theory-based models with ML) promises improved predictions.
    \item I showed an application using gene expression (Bgee).
    \item Adding our mechanistic predictions (based on GEESE) boosted quality
\end{itemize}
}}

\end{frame}

\begin{frame}{Discussion (cont.)}
    \begin{itemize}[<+->]
        \item This is the core of an R01 first submitted in Feb 2022.
        \item The original version did not feature any ML, just the mechanistic part.
        \item Two important critiques: ``Not leveraging large genomic data'' and ``Not using Neural Networks''
        \item We believe we are addressing both using gene expression data (Bgee) and Mechanistic ML.
        \item ...your thoughts?
    \end{itemize}
\end{frame}

\begin{frame}{}
\begin{center}
    \Large Thank you!
\end{center}
\maketitle
\end{frame}

\appendix

\begin{frame}[allowframebreaks]{References}
    \printbibliography
\end{frame}

%-------------------------------------------------------------------------------
\begin{frame}[c]
\frametitle{Tree likelihoods: Felsenstein's Pruning algorithm}
\begin{center}
    \mode<beamer>{%
    \includegraphics<1>[width=.6\linewidth]{aphylo-equation1.pdf}%
    \includegraphics<2>[width=.6\linewidth]{aphylo-equation.pdf}
    }%
    \mode<handout>{%
    \includegraphics[width=.6\linewidth]{aphylo-equation.pdf}
    }
\end{center}
\vfill\hfill\uncover<2->{... I implemented this (and more) on \textbf{barry}}
\end{frame}

\begin{frame}{Some computational features of \textbf{barry}}
\begin{figure}
    \centering
    \includegraphics[width=1\textwidth]{fig/barry-computing.png}
    \label{fig:barry}
\end{figure}
\end{frame}

\end{document}